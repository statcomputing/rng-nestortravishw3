\documentclass[]{article}
\usepackage{lmodern}
\usepackage{amssymb,amsmath}
\usepackage{ifxetex,ifluatex}
\usepackage{fixltx2e} % provides \textsubscript
\ifnum 0\ifxetex 1\fi\ifluatex 1\fi=0 % if pdftex
  \usepackage[T1]{fontenc}
  \usepackage[utf8]{inputenc}
\else % if luatex or xelatex
  \ifxetex
    \usepackage{mathspec}
  \else
    \usepackage{fontspec}
  \fi
  \defaultfontfeatures{Ligatures=TeX,Scale=MatchLowercase}
\fi
% use upquote if available, for straight quotes in verbatim environments
\IfFileExists{upquote.sty}{\usepackage{upquote}}{}
% use microtype if available
\IfFileExists{microtype.sty}{%
\usepackage{microtype}
\UseMicrotypeSet[protrusion]{basicmath} % disable protrusion for tt fonts
}{}
\usepackage[margin=1in]{geometry}
\usepackage{hyperref}
\hypersetup{unicode=true,
            pdftitle={Homework 3},
            pdfborder={0 0 0},
            breaklinks=true}
\urlstyle{same}  % don't use monospace font for urls
\usepackage{graphicx,grffile}
\makeatletter
\def\maxwidth{\ifdim\Gin@nat@width>\linewidth\linewidth\else\Gin@nat@width\fi}
\def\maxheight{\ifdim\Gin@nat@height>\textheight\textheight\else\Gin@nat@height\fi}
\makeatother
% Scale images if necessary, so that they will not overflow the page
% margins by default, and it is still possible to overwrite the defaults
% using explicit options in \includegraphics[width, height, ...]{}
\setkeys{Gin}{width=\maxwidth,height=\maxheight,keepaspectratio}
\IfFileExists{parskip.sty}{%
\usepackage{parskip}
}{% else
\setlength{\parindent}{0pt}
\setlength{\parskip}{6pt plus 2pt minus 1pt}
}
\setlength{\emergencystretch}{3em}  % prevent overfull lines
\providecommand{\tightlist}{%
  \setlength{\itemsep}{0pt}\setlength{\parskip}{0pt}}
\setcounter{secnumdepth}{0}
% Redefines (sub)paragraphs to behave more like sections
\ifx\paragraph\undefined\else
\let\oldparagraph\paragraph
\renewcommand{\paragraph}[1]{\oldparagraph{#1}\mbox{}}
\fi
\ifx\subparagraph\undefined\else
\let\oldsubparagraph\subparagraph
\renewcommand{\subparagraph}[1]{\oldsubparagraph{#1}\mbox{}}
\fi

%%% Use protect on footnotes to avoid problems with footnotes in titles
\let\rmarkdownfootnote\footnote%
\def\footnote{\protect\rmarkdownfootnote}

%%% Change title format to be more compact
\usepackage{titling}

% Create subtitle command for use in maketitle
\newcommand{\subtitle}[1]{
  \posttitle{
    \begin{center}\large#1\end{center}
    }
}

\setlength{\droptitle}{-2em}
  \title{Homework 3}
  \pretitle{\vspace{\droptitle}\centering\huge}
  \posttitle{\par}
  \author{}
  \preauthor{}\postauthor{}
  \date{}
  \predate{}\postdate{}


\begin{document}
\maketitle

\section{2. Given probability densities f(x) and
g(x)}\label{given-probability-densities-fx-and-gx}

\begin{enumerate}
\def\labelenumi{\alph{enumi})}
\item
  Find normailizing constant C, and show g(x) is a mixture of two gamma
  functions and determine weights

\begin{verbatim}
 Set theta =1 to find constant C
 fun <- function(x) {((x^.5)*exp(-x) + 2*exp(-x))} # distribute exp(-x) to separate functions
 integrate(fun, lower = 0, upper = Inf)
 #receive 2.886227
 C     <- 1 / 2.886227        #normalizing constant
\end{verbatim}

  As g(x) can be separated into \(exp(-x) * (2*x^{(\theta -1)}\)\$ and
  \[x ^ (\theta - .5) * e^{(-x)} \] we can clearly see g is a
  composition of gamma functions where \[r = 1\] and \[a = \theta\]
\end{enumerate}

\begin{verbatim}
  fun.1     <- function(x) {2*exp(-x)}
  Fun.1     <- integrate(fun.1, lower = 0, upper = Inf)
  fun.2     <- function(x) {x^(.5)*exp(-x)}
  Fun.2     <- integrate(fun.2, lower = 0, upper = Inf)
  
weightfun.1 <- print((Fun.1$$value)*C)        #weighted component 1
weightfun.2 <- print((Fun.2$$value)*C)        #weighted component 2
\end{verbatim}

\begin{enumerate}
\def\labelenumi{\alph{enumi})}
\setcounter{enumi}{1}
\item
\end{enumerate}

\begin{verbatim}
probdenG <- function(x, theta) {
            (2 * x ^ (theta-1) + x ^ (theta - .5)) * exp(-x)
}
 
\end{verbatim}

\begin{enumerate}
\def\labelenumi{\arabic{enumi})}
\setcounter{enumi}{1}
\item
  \begin{enumerate}
  \def\labelenumii{\alph{enumii})}
  \tightlist
  \item
    As \[g(x) = (2x^{\theta-1}+x^{\theta-1/2})e^{-x}\]
  \end{enumerate}
\end{enumerate}

we find normalizing constant C such that
\[2C*\Gamma(\theta)+C*\Gamma(\theta+\frac{1}{2}=1)\]
\[\Rightarrow C=\frac{1}{2\Gamma(\theta)+\Gamma(\theta+\frac{1}{2})}\]
Therefore we can change \(g(x)\) to
\[g(x)=\frac{1}{2\Gamma(\theta)+\Gamma(\theta+\frac{1}{2})}2x^{\theta-1}e^{-x}+\frac{1}{2\Gamma(\theta)+\Gamma(\theta+\frac{1}{2})}2x^{\theta-1/2}e^{-x}\]

\[\Rightarrow g(x) = \Gamma (\theta, 1)\] weighted by
\[2 * \Gamma(\theta) / (2\Gamma(\theta)+\Gamma(\theta+.5))\] and
\[ \Gamma (.5*\theta, 1)\] weighted by and
\[\Gamma(\theta+.5) / (2 * \Gamma(\theta)+\Gamma(\theta+.5))\]

\begin{enumerate}
\def\labelenumi{\alph{enumi})}
\setcounter{enumi}{1}
\tightlist
\item
  ``` fun.g \textless{}- function(theta) \{ n \textless{}- 10000 weight
  \textless{}- 2\emph{gamma(theta) / (2}gamma(theta)+gamma(theta+1/2))
  iter \textless{}- 0 niter \textless{}- c() rand \textless{}- runif(n)
  for (i in 1:n) \{ if (rand{[}i{]} \textless{} weight) \{ x
  \textless{}- rgamma91, theta, 1) iter \textless{}- iter + 1 niter
  \textless{}- c(niter, x) \} \} return(niter) \}
\end{enumerate}

graph.g \textless{}- fun.g(1) hist(graph.g) ```


\end{document}
